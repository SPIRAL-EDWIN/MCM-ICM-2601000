\documentclass{mcmthesis}
\mcmsetup{CTeX = false,   % 使用 CTeX 套装时,设置为 true
        tcn = 1111111, problem = ABCDEF,
        sheet = true, titleinsheet = true, keywordsinsheet = true,
        titlepage = false, abstract = true}
\usepackage{palatino}
\usepackage{lipsum}
\usepackage{algorithm}
\usepackage{algpseudocode}
\usepackage{subfig}
\usepackage{mdframed}
\usepackage{threeparttable}
\usepackage{indentfirst}
\usepackage{siunitx}
\usepackage{pifont}
\usepackage[T1]{fontenc}
\usepackage{newtxtext,newtxmath}
\usepackage[most]{tcolorbox}
\usepackage{makecell}
\usepackage{multirow}
\usepackage{xurl}
\sisetup{
  table-number-alignment = center,
  round-mode = places,
  round-precision = 2
}

\setlength{\parindent}{2em}
\makeatletter
\renewcommand\@cite[1]{\textsuperscript{[#1]}}
\makeatother

\title{Cracking the Cyber - Puzzle: KDMF in Action}
% 这一段是备忘录部分,如果题目没有让写备忘录 或者书信 可以不要
% \author{\small \href{http://www.latexstudio.net/}
%   {\includegraphics[width=7cm]{mcmthesis-logo}}}
% \date{\today}

%  \memoto{\LaTeX{}studio}
% \memofrom{Liam Huang}
% \memosubject{Happy \TeX{}ing!}
% \memodate{\today}
% %\memologo{\LARGE I'm pretending to be a LOGO!}

\begin{document}  
\begin{abstract}
  
In the digital era, cybercrime has become a significant global concern due to the increasing 
interconnectedness facilitated by modern technology. This study aims to identify patterns for data 
- driven development and refinement of national cybersecurity policies. Specifically, we developed
the \textbf{KDMF} model, consisting of three sub - models, to achieve this goal.

To analyze \textbf{the global distribution of cybercrime}, we selected indicators like the National 
Cyber Security Index (NCSI), Cybersecurity Exposure Index (CEI), Legal Measures Index, and 
Cybercrime Reporting Rate. Through descriptive analysis and \textbf{K-means clustering}, we found 
that countries with high NCSI scores and strong legal measures, such as those in the Purple Cluster, 
are better at preventing and prosecuting cybercrimes. In contrast, countries in the Yellow Cluster 
with low scores are high - risk targets. 

To assess \textbf{the effectiveness of national cybersecurity policies}, we employed Latent Di-
richlet Allocation (\textbf{LDA}) and the Difference - in - Differences (\textbf{DID}) model. LDA categorized 
policies into six themes, and DID analysis showed that Data Protection and Privacy Regulations, 
Cybersecurity Obligations of Network Operators, and Cross - Border Data Flow Rules are effec-
tive in reducing cybercrime rates, while Cybercrime Criminal Legislation, Critical Information 
Infrastructure Protection, and Cybersecurity Incident Emergency Response Mechanisms need im-
provement. 

To identify \textbf{the correlation between demographic factors and cybercrime distribution}, 
we used the Maximal Information Coefficient (\textbf{MIC}) method and \textbf{factor analysis}. The results 
indicated that the Internet Penetration Rate, Active mobile - broadband subscriptions, and GDP 
per capita growth have relatively strong correlations with cybercrime, while GDP and tertiary ed-
ucation enrollment have weaker effects. Different variables also have varying impacts across coun-
tries, which is similar to the pattern found when analyzing the distribution. 
Finally, we arranged our research findings into a 1 - page memo for country leaders attending 
the ITU Summit on Cybersecurity, trying to offer key findings relevant for policymakers to better 
address cybercrime challenges. 

Sensitivity analysis on the DID and MIC models indicates the models' stability and reliability. 
Overall, this research provides valuable insights for policymakers, helping them understand cy-
bercrime patterns, evaluate policy effectiveness, and consider demographic factors when formu-
lating more targeted and effective cybersecurity policies. 

\begin{keywords}
Cybercrime, KDMF, K-means, DID, MIC, Policy
\end{keywords}
\end{abstract}

\maketitle
\tableofcontents
  \thispagestyle{empty}
  \newpage

\section{Introduction}
\subsection{Problem Background}
In the digital age, rapid technological progress has connected the world closely. However,
this has led to a sharp increase in cybercrime. Reports show that the annual global cost of cyber-
crime is expected to reach a staggering \$10.5 trillion by 2025\cite{1}. High - profile cases, like the 2024
cyber - attacks on cryptocurrency exchanges causing hundreds of millions of dollars in losses,
clearly demonstrate the severe financial impact.
Cybercrime's transnational nature complicates matters. A single case can involve multiple
countries, making it difficult to enforce laws due to jurisdictional issues. Moreover, many organi-
zations hide cyber - attacks. A study found that over 60\% of small - and medium - sized enterprises
in a certain region didn't report security breaches, hiding the true scale of cyber threats.
Against this backdrop, the development and refinement of national cybersecurity policies
grounded in data - driven insights are of paramount importance. Comprehending the patterns of
cybercrime distribution, evaluating the efficacy of existing policies, and discerning the correlations
with national demographics can offer invaluable guidance. Such knowledge equips countries to
formulate more robust policies, foster closer international cooperation, and safeguard their digital
assets and national security more effectively.
\begin{figure}[h]
  \small
  \centering
  \includegraphics[width=8cm]{figure0.png}
  \caption{Top 10 threats in the Emerging Cybersecurity Threats 2030 published by the EU} \label{fig:aa}
\end{figure}


\subsection{Restatement of the Problem}
The swift dissemination of modern technology has augmented global connectivity but has
also precipitated a dramatic upsurge in cybercrime. This phenomenon has imposed formidable
challenges on countries in their endeavors to safeguard their digital ecosystems. To extract patterns
from effective national cybersecurity policies and laws and to propel the data - driven evolution
and improvement of these policies, the following undertakings are imperative.

\begin{itemize}[\ding{226}]
  \item \textbf{Task 1:} Analyze the global distribution of cybercrime. Identify high - target countries,successful and foiled crime regions, and areas of reporting and prosecution. Explore underlying patterns.
  \item \textbf{Task 2:} Examine national cybersecurity policies. Compare with cybercrime distribution to find effective or ineffective policy components. Consider policy adoption time.
  \item \textbf{Task 3:} Investigate the correlation between demographic factors like internet access, wealth, and education levels and cybercrime distribution. Analyze their impact on the theory of effective policies.
  \item \textbf{Task 4:} Based on data quantity, quality, and reliability, identify limitations for policy-makers when using research findings for policy development.
  \item \textbf{Task 5:} Write a one - page memo for country leaders at the ITU Cybersecurity Summit.Provide a non-technical overview of work objectives, context, theory, and key findings.
\end{itemize}

\subsection{Our Work}
Our work mainly includes developing the \textbf{KDMF} model to drive data - informed national
cybersecurity policies.
For \textbf{Task 1}, we collected cybercrime data, visualized it, and used \textbf{K-Means clustering} to
understand global cybercrime distribution.
For \textbf{Task 2}, we gathered and analyzed national cybersecurity policies. We categorized poli-
cies into six themes via \textbf{LDA}, and evaluated their effectiveness with \textbf{DID} model.
For \textbf{Task 3}, we used the \textbf{MIC} and \textbf{factor analysis} to explore the correlation between de-
mographics and cybercrime.
These efforts provided insights into how demographic factors influence cybercrime, guiding
more targeted cybersecurity strategies.
\begin{figure}[h]
  \small
  \centering
  \includegraphics[width=9cm]{figure1.png}
\end{figure}
\section{Assumptions and Justifications}  
\begin{itemize}[\ding{226}]
\item \textbf{Assumption 1:} Data Completeness and Representativeness. The 9,867 cyber - crime
cases sourced from the VCDB platform are assumed to be sufficiently comprehensive
for representing global cyber - crime distribution.\cite{2} Similarly, the cyber - security - re-
lated indicators used in our analysis are assumed to be authoritative.

\textbf{Justification:} VERIS -- it is rooted in the examination of evidence and post-incident
analysis, and given that these databases are exclusively from websites of global organi-
zations, it's logical to deduce that the quality of their data is superior.

\item \textbf{Assumption 2:} Stable Socio - Economic and Policy Context. We assume that the national
economic conditions and policy environments remain relatively stable during the study
period. Drastic economic events like recessions or financial crises, as well as sudden,
unforeseen policy overhauls (beyond those being studied), are not expected.

\textbf{Justification:} Economic instability can significantly impact cyber - crime rates. Unstable
policies would make it challenging to establish a clear link between policy implementa-
tion and cyber - crime distribution. A stable context allows us to isolate the effects of the
policies and other variables we are analyzing.

\item \textbf{Assumption 3:} Causal Independence of Variables. National demographics such as inter-
net access, wealth, and education levels are assumed to be causally independent of policy
implementation, except for the relationships we are explicitly exploring.

\textbf{Justification:} If demographics and policy implementation are causally intertwined in
unaccounted ways, it would be challenging to disentangle the individual effects of each
on cyber - crime distribution. By assuming causal independence, we can more easily
analyze the correlations and test our theories.

\end{itemize}
\section{Notations}
The key mathematical notations used in this paper (mostly in the DID Method) are listed in Table 1. The data of different dimensions are normalized in the concrete operation process
 \begin{table}[H]
  \caption{Notations used in this paper}
    \centering
    \begin{tabular}{cc}
        \toprule
        Symbol & Description \\
        \midrule
        $Y_{it}$ & The number of cybercrimes in country $i$ during year $t$ \\
        $\alpha$ & Intercept; baseline cyber - crime level when all independent variables are zero  \\
        $\beta_{1}$ & Difference in cyber - crime rates pre - policy implementation \\
        $\beta_{2}$ & General time - trend change in cyber - crime rates across all countries \\
        $\beta_{3}$ & DID estimator; net policy implementation effect on cyber - crime rates \\
        $X_{kit}$ & k-th normalized control variable for country $i$ in year $t$ \\
        $\gamma_k$ & Coefficient of $X_{kit}$  \\
        $\epsilon_{it}$ & Error term \\ 
        \bottomrule
    \end{tabular}
  \end{table}


\section{K-Means approach: Unveiling the Tapestry of Cybercrime}
\subsection{Introduction}
With development of technology, cybercrime has transcended borders, emerging as a perva-
sive threat that affects nations worldwide. Understanding its global distribution is essential for
developing effective strategies to combat this menace. This part conducts a comprehensive analy-
sis using key indicators and a clustering approach to reveal patterns in cybercrime distribution.

\subsection{Indicators Selection}
To conduct an in-depth examination of the global distribution of cybercrime, we selected four
key indicators that collectively illuminate various dimensions of cybercrime dynamics and national
responses:

\begin{itemize}[\ding{226}]
  \item  \textbf{National Cyber Security Index (NCSI):} This index measures a country's readiness and
capability to prevent cyber threats and manage cyber incidents.\cite{3} It reflects the robust-
ness of a nation's cybersecurity infrastructure, policies, and strategic capacities. High
NCSI scores indicate effective cybersecurity measures, reducing vulnerability to cyberat-
tacks. Conversely, low scores suggest inadequate defenses.
  \item \textbf{Cybersecurity Exposure Index (CEI):} The CEI evaluates the level of exposure a coun-
try has to cyber risks by considering both the frequency of cyberattacks and the country's
cybersecurity commitment.\cite{4} A high CEI score indicates significant exposure, poten-
tially due to frequent attacks and insufficient cybersecurity measures.
  \item \textbf{Legal Measures (Global Cybersecurity Index):} This indicator assesses the compre-
hensiveness and effectiveness of a country's laws, regulations, and enforcement mecha-
nisms related to cybercrime.\cite{5} Strong legal measures can deter potential cybercriminals
through the threat of legal repercussions and provide the means to prosecute and penalize
offenders.
  \item \textbf{Cybercrime Reporting Rate:} This metric reflects the frequency with which cybercrime
incidents are reported in a country relative to its number of internet users. High reporting
rates suggest public awareness of cyber threats, efficient reporting mechanisms, and trust
in law enforcement.     
\end{itemize}

By examining cybersecurity preparedness, exposure to threats, legal infrastructures, and re-
porting behaviors, we gain comprehensive insights into both the vulnerabilities that enable cyber-
crime and the capacities that enable nations to respond effectively.  
\subsection{Data Analysis}
\subsubsection{Descriptive Analysis of Individual Indicators}
\begin{enumerate}
  \item National Cyber Security Index (NCSI)
The NCSI is a critical measure of a nation's ability to protect itself against cyber threats,
evaluating factors such as cybersecurity policies, education, incident response capabilities, and
protection of critical infrastructure.

  \begin{itemize}[\ding{226}]
    \item \textbf{High-Scoring Countries:} Nations such as Belgium (94.81), Lithuania (93.51), and Germany
(90.91) exhibit excellent preparedness. Their high NCSI scores reflect comprehensive cyber-
security strategies, well-established policies, continuous investment in cybersecurity infra-
structure, and robust incident response mechanisms. These countries are better positioned to
prevent cyberattacks and mitigate their impact, reducing the likelihood of successful cyber-
crimes.

  \item \textbf{Low-Scoring Countries:} In contrast, countries like Afghanistan (11.69), Saudi Arabia
(10.39), and Myanmar (10.39) have significantly lower NCSI scores. This indicates insuffi-
cient cybersecurity measures, outdated or absent cybersecurity policies, limited resources, and
lack of awareness or training. Such vulnerabilities make them attractive targets for cybercrim-
inals, who can exploit these weaknesses to carry out attacks with a higher chance of success.
    \begin{figure}[H]
      \centering
      \includegraphics[width=12cm]{figure2.png}
      \caption{Which countries are disproportionately high targets of cybercrimes?}
    \end{figure}  
  \end{itemize}

 \item Cybersecurity Exposure Index (CEI)
To provide a more intuitive representation of countries' cybersecurity levels, we processed
the original Cybersecurity Exposure Index (CEI) values. Given that the original CEI ranges from
0 to 1, with higher scores indicating higher exposure to cybercrime, we transformed the data by
calculating $(1 - original CEI value) \times   100$. This new calculation results in a modified CEI
where a higher score now corresponds to a lower exposure level. This transformation was carried
out to simplify the interpretation of the index, making it easier to understand which countries have
a lower vulnerability to cybercrime at a glance.

  \begin{itemize}[\ding{226}]
    \item \textbf{High-Security Countries:} Finland (89.0) and Belgium (81.0) have high CEI scores, which
may face lower exposure levels. These countries have well - developed digital infrastructure
and high internet penetration, which would seemingly make them vulnerable. However, their
advanced security strategies, including strong legal frameworks, active incident response
teams, and high - level security awareness, effectively reduce their exposure to cyber threats,
resulting in low CEI scores.

  \item \textbf{Low-Security Countries:} Afghanistan (0.0) and Myanmar (0.0) report low CEI scores, sug-
gesting they face a substantial number of cyber threats. Their limited technological capabili-
ties and underdeveloped digital infrastructure result in less effective security measures. This 
makes them more vulnerable to cyber threats, thereby increasing their exposure. Moreover,
weak legal frameworks related to cybercrime in these nations are ineffective at deterring malicious actors.
\end{itemize}
  \begin{figure}[H]
      \centering
      \includegraphics[width=12cm]{figure3.png}
      \caption{Where are cybercrimes successful, and where are they thwarted?}
    \end{figure}  
  
\item Legal Measures (Global Cybersecurity Index)
Legal frameworks are fundamental in combating cybercrime, establishing the legal basis for
action against cybercriminals.
\begin{itemize}[\ding{226}]

  \item \textbf{Countries with Strong Legal Measures:} Nations such as Germany (20) and Australia (20)
possess comprehensive cyber laws, regulations, and enforcement mechanisms. They have
specific legislation addressing cybercrime, enforce legal actions against offenders, and partic-
ipate in international cooperation. This legal rigor deters cybercriminals and enables effective
prosecution when crimes occur.
  \item \textbf{Countries with Weak Legal Measures:} Afghanistan (5.81) and Turkey (10.12) have lower
scores, indicating insufficient legal frameworks to address cybercrime effectively. Weak or
outdated laws, lack of enforcement, and judicial inefficiencies hinder their ability to prosecute
cybercriminals, potentially making these countries safer havens for such activities.
\end{itemize}
  \begin{figure}[H]
      \centering
      \includegraphics[width=12cm]{figure4.png}
      \caption{Where are cybercrimes prosecuted?}
    \end{figure}

\item Cybercrime Reporting Rate
Reporting plays a crucial role in understanding the scale of cybercrime and implementing
appropriate responses.
\begin{itemize}[\ding{226}]
  \item \textbf{High Reporting Rates:} The United States (2.24e-05) and the United Kingdom (8.88e-06)
exhibit relatively high reporting rates, reflecting efficient reporting systems, high public
awareness, and trust in authorities. These countries encourage reporting through accessible
channels, public education campaigns, and assurances of support and confidentiality. High
reporting rates enable authorities to gather intelligence, track trends, and allocate resources
effectively.

  \item \textbf{Low Reporting Rates:} Countries like India, Ukraine, and Uganda display minimal or zero
reporting rates. This may result from cultural stigmas, lack of awareness about cybercrime,
mistrust of authorities, or insufficient reporting mechanisms. Low reporting hinders the ability
of law enforcement and policymakers to grasp the true extent of cybercrime, allowing threats
to persist unaddressed.
\end{itemize}
  \begin{figure}[H]
      \centering
      \includegraphics[width=12cm]{figure5.png}
      \caption{Where are cybercrimes reported?}
    \end{figure}
\end{enumerate}
\subsubsection{Clustering Analysis of Cybercrime Distribution Patterns}
\begin{enumerate}
  \item Clustering Process Overview
To uncover underlying patterns in the global distribution of cybercrime, we employed K-
means clustering,\cite{6} focusing on the NCSI and Legal Measures indicators. This approach groups
countries with similar characteristics, enabling the identification of clusters with shared vulnera-
bilities or strengths.
  \begin{itemize}[\ding{226}]
  \item \textbf{Data Standardization:} Ensured comparability across different scales and units.
  \item \textbf{Determination of Optimal Clusters:} Used the Elbow Method to identify the optimal number
of clusters for the K-means algorithm. It involved calculating inertia (sum of squared distances
to nearest centers) for different K values. The Elbow Method graph showed inertia decrease
slowed around K = 3, the "elbow" point, meaning more clusters above K = 3 had diminishing
quality returns. We also considered the Silhouette Score. Although K = 2 had the highest score
for cohesion and separation, overall, K = 3 balanced inertia reduction and high Silhouette
Score better, so it was chosen as optimal.     
    \begin{figure}[H]
      \centering  
      \includegraphics[width=12cm]{figure6.png}
      \caption{Elbow and Silhouette Plots for Cluster Number Selection}
    \end{figure}
 \item \textbf{K-Means Clustering:} Classified countries into three clusters---Purple, Yellow, and Cyan---
based on similarities in cybersecurity preparedness and legal measures. The Purple Cluster
includes countries that exhibit high standardized values for both NCSI and Legal Measures,
signifying strong cybersecurity preparedness and well - established legal frameworks. In con-
trast, the Yellow Cluster consists of countries with low standardized values for these indica-
tors, indicating significant vulnerabilities in both cybersecurity and legal protection. The Cyan
Cluster represents countries with a more heterogeneous set of characteristics, where there are
a mix of strengths and weaknesses in either NCSI or Legal Measures.
 \item \textbf{Visualization and Interpretation:} Mapped the clustering results to facilitate analysis. Using
principal component analysis (PCA) to project high - dimensional data onto a 2D plane helped
visualize clustering results. Each country was a point, and cluster centers were red "X" sym-
bols. This showed country distribution in clusters and overlapping areas, highlighting the
complexity of differentiating countries by just NCSI and Legal Measures. It also showed a
country's cyber - security was affected by many factors, calling for a more comprehensive
analysis with more cyber - crime related variables.
\end{itemize}
  \begin{figure}[H]
      \centering
      \includegraphics[width=8cm]{figure7.png}
      \caption{PCA - based 2D Visualization of Cyber - security Clustering Results}
    \end{figure}
  
\item Analysis of Clustering Results
  \begin{tcolorbox}[
    colback=purple!10,      % 背景色(浅紫)
    colframe=purple!30,     % 边框色
    arc=6pt,                % 圆角
    boxrule=0pt,            % 去掉边框线
    left=8pt, right=8pt,    % 内边距
    top=4pt, bottom=4pt,
    drop shadow,            % 阴影
  ]
    Purple Cluster: Prepared and Resilient Nations
  \end{tcolorbox}

  \textbf{Characteristics:} High standardized values for both NCSI and Legal Measures.

  \textbf{Examples:} Germany, Belgium, Australia, United Kingdom.

  \textbf{Analysis:} Countries in the Purple Cluster exhibit strong cybersecurity infrastructures and compre-
hensive legal frameworks. Their high NCSI scores reflect advanced technical capabilities, effective policies, and proactive strategies.
Robust legal measures indicate stringent laws, effective enforcement, and active participation in international cooperation against cybercrime.

  \textbf{Outcomes:} These nations are better at preventing cyberattacks and successfully thwarting cyber
criminal activities. High reporting rates facilitate timely responses to threats. Strong legal systems
enable effective prosecution of cybercriminals, deterring potential offenders.

  \begin{tcolorbox}[
    colback=Cyan!70,      
    colframe=Cyan!100,     
    arc=6pt,               
    boxrule=0pt,           
    left=8pt, right=8pt,   
    top=4pt, bottom=4pt,
    drop shadow,              
  ]
    Cyan Cluster: Mixed Preparedness and Resilience
  \end{tcolorbox}


  \textbf{Characteristics:} Moderate or mixed standardized values.

  \textbf{Examples:} India, Ukraine, Finland.

  \textbf{Analysis:} Countries in the Cyan Cluster display a mix of strengths and weaknesses. Some may
  have strong cybersecurity measures but weaker legal frameworks, or vice versa. This inconsistency
  may be due to transitional economies, evolving policies, or resource constraints.

  \textbf{Outcomes:} These nations experience variable success in preventing and managing cybercrime.
While some cyberattacks may be thwarted, others succeed due to gaps in defenses or enforcement.
Reporting and prosecution rates may be inconsistent, reflecting the uneven development of sys-
tems and awareness.

  \begin{tcolorbox}[
    colback=yellow!70,      
    colframe=yellow!100,     
    arc=6pt,               
    boxrule=0pt,           
    left=8pt, right=8pt,   
    top=4pt, bottom=4pt,
    drop shadow,              
  ]
    Yellow Cluster: High-Risk and Vulnerable Countries
  \end{tcolorbox}
  \textbf{Characteristics:} Low standardized values for both NCSI and Legal Measures.

  \textbf{Examples:} Afghanistan, Myanmar, Saudi Arabia, Turkey.

  \textbf{Analysis:} The Yellow Cluster comprises countries with significant vulnerabilities due to inade-
quate cybersecurity preparedness and weak legal measures. These nations lack the infrastructure,
policies, and capabilities necessary to defend against cyber threats effectively. Legal frameworks
may be outdated, poorly enforced, or insufficient to address the complexities of cybercrime.

  \textbf{Outcomes:} Such countries are disproportionately high targets for cybercriminals, who exploit
these weaknesses. Cybercrimes are more likely to succeed and remain unreported or unprosecuted.
The lack of deterrence emboldens cybercriminals, perpetuating a cycle of vulnerability.

\item  Integration with Individual Indicator Analyses
  
The clustering results corroborate the findings from the individual indicator analyses:
  \begin{itemize}[\ding{226}]
    \item Correlation Between Preparedness and Vulnerability: Countries with high NCSI scores tend
to have strong legal measures, suggesting a holistic approach to cybersecurity that includes
both technological defenses and legal deterrents.
    \item Impact on Cybercrime Outcomes: The clusters illustrate how variations in preparedness and
legal frameworks influence the prevalence and success of cybercrime. The Purple Cluster's
strengths result in lower rates of successful cyberattacks and higher rates of reporting and
prosecution. Conversely, the Yellow Cluster's weaknesses lead to higher vulnerability and
fewer resources to combat cyber threats effectively.
    \item Regional Patterns: The clustering reveals geographical trends, with many developing nations
falling into the Yellow Cluster, while more developed nations generally occupy the Purple
Cluster. This highlights the role of economic development and resource allocation in cyber-
security readiness.
    \end{itemize}
\end{enumerate}

\subsection{Discussion}
\subsubsection{Synthesis of Findings}
\begin{enumerate}
  \item  Cybercrime Distribution and Targeting
  \begin{itemize}[\ding{226}]
    \item \textbf{High-Risk Targets:} Countries in the Yellow Cluster are disproportionately targeted by cy-
bercriminals due to their low cybersecurity preparedness and weak legal measures. These vul-
nerabilities result in higher success rates of attacks.
    \item \textbf{Successful and Thwarted Cybercrimes:} In the Purple Cluster, robust cybersecurity infra-
structures and stringent legal frameworks enable effective prevention and mitigation of cy-
bercrimes. Cyberattacks are more likely to be detected, reported, and thwarted. Conversely,
the Yellow Cluster experiences higher rates of successful cybercrimes due to their vulnera-
bilities.
  \end{itemize}
  \item  Reporting and Prosecution of Cybercrimes
  \begin{itemize}[\ding{226}]
    \item \textbf{Reporting Rates:} High reporting rates in the Purple Cluster reflect public aware-
ness, efficient reporting mechanisms, and trust in law enforcement. This transparency allows
for better tracking of cyber threats and more effective responses. In the Yellow and Cyan
Clusters, underreporting due to lack of awareness, inadequate infrastructure, or distrust ham-
pers efforts to combat cybercrime.
    \item \textbf{Prosecution Efficacy:} Strong legal frameworks in the Purple Cluster enable effective pros-
ecution of cybercriminals, acting as a deterrent. Weak legal measures in the Yellow Cluster
result in challenges in prosecuting offenders, allowing cybercrime to persist with minimal
consequences.
  \end{itemize}
\end{enumerate}
\subsubsection{Emergent Patterns and Implications}
\begin{enumerate}
  \item \textbf{Interplay Between Preparedness and Vulnerability:} There is a clear inverse relationship be-
tween a country's cybersecurity preparedness and its vulnerability to cybercrime. Nations investing
in cybersecurity measures and legal reforms are less susceptible to attacks and better at managing
incidents.
  \item \textbf{Significance of Legal Frameworks:} Effective legal measures are crucial for deterring cyber-
criminals and prosecuting offenders. International cooperation and adherence to global cybersecu-
rity norms enhance a country's ability to combat cybercrime.
  \item \textbf{Role of Societal Factors:} Public awareness and education significantly impact reporting rates.
Cultural attitudes towards technology and authority influence the willingness to report incidents.
  \item \textbf{Economic and Developmental Factors:} Developing nations often lack resources to invest in
cybersecurity infrastructure or legal reforms. Economic disparities contribute to the uneven distri-
bution of cybercrime impacts globally.
\end{enumerate}

\section{DID Analysis: Assessing Cyber - security Policies}

In our prior exploration of the global cybercrime distribution, we grasped the gravity and
complexity of cyber threats. Now, we turn to assessing national cyber - security policies' effec-
tiveness. Cybercrime's rapid growth in the digital age demands a close look at these policies.

We use Latent Dirichlet Allocation (LDA) \cite{7} to categorize various national security policies,
and Difference - in - Differences (DID) model \cite{8} to analyze their impact on cybercrime rates. By
comparing policies with cybercrime distribution, we aim to find which policy elements are effec-
tive or not in preventing, prosecuting, or mitigating cybercrimes, also considering policy adoption
timings.
\subsection{Cracking the Policy Code: LDA and DID in Action}
\subsubsection{LDA - based Thematic Analysis}

In the complex landscape of cybersecurity, national security policies vary widely in content
and focus. To distill meaningful insights from these policies, we turned to Latent Dirichlet Allo-
cation (LDA), a powerful topic - modeling technique in natural language processing.

Our approach began with collecting and pre - processing a diverse set of national cybersecu-
rity policies. This involved basic text cleaning, including the removal of stop - words that carry
little semantic value. We then fed the pre - processed data into the LDA model. Through iterative
experiments and evaluations using perplexity and coherence scores, we determined the optimal
number of latent topics within the policies. This allowed us to categorize the policies into distinct
themes, which served as the foundation for further analysis using the Difference - in - Differences
(DID) model to assess the impact of each policy theme on cybercrime rates.
The six topics we identified are as follows:

\begin{itemize}[\ding{226}]
 \item \textbf{Cybercrime Criminal Legislation:} This topic encompasses keywords like "Cyber-
crime", "warrant", "Penalty", etc. It focuses on the legal frameworks and punitive
measures in place to combat cybercrimes. Understanding these laws is crucial for as-
sessing how countries prosecute cybercriminals.
 \item \textbf{Data Protection and Privacy Regulations:} With keywords such as "Personal", "data", "En-
cryption", this topic is centered around how countries safeguard personal data. In an era of
rampant data - driven cybercrimes, these regulations play a vital role in preventing identity
theft, data breaches, and other privacy - related cyber threats.
 \item \textbf{Cybersecurity Obligations of Network Operators:} Keywords like "Cybersecurity", "oper-
ator" define this topic. It delves into the responsibilities and requirements placed on network
operators to ensure the security of their networks, which is a key aspect in preventing cyberat-
tacks at the network level.
 \item \textbf{Critical Information Infrastructure Protection:} Covering keywords "Infrastructure", "Pro-
tection", "obligation", this topic is about safeguarding the critical information infrastructure
of a country. These infrastructures are the backbone of a nation's digital ecosystem, and their
protection is essential for national security.
 \item \textbf{Cybersecurity Incident Emergency Response Mechanisms:} This topic, marked by key-
words "security", "warning", "Compliance", deals with how countries respond to cyber - se-
curity incidents. An effective response mechanism can mitigate the damage caused by cyber-
crimes.
 \item \textbf{Cross - Border Data Flow Rules:} Keywords "border", "exporter", "consent" define this topic.
In a globalized digital world, cross - border data flow is common, and these rules govern how
data is transferred across national boundaries, which is crucial for preventing cross - border
cybercrimes.
\end{itemize}
\begin{figure}[H]
      \centering
      \includegraphics[width=14cm]{figure8.png}
      \caption{Word Clouds Representing Six Cybersecurity Policy Themes}
    \end{figure}

\subsubsection{Efficiency Analysis by DID}

After categorizing the policies using LDA, we employed the Difference - in - Differences
(DID) approach to quantify the causal impact of each policy theme on cybercrime rates.
\begin{enumerate}
  \item \textbf{Data Preparation}\par

\leavevmode\hspace*{1.5em} We meticulously selected countries for the experimental and control groups. For each policy
theme, the experimental group consisted of 1 - 2 countries that had implemented the relevant pol-
icies. In contrast, the control group was composed of an equal number of countries with similar
internet penetration rates but without such policies. This careful selection was crucial for isolating
the treatment effect of the policies, ensuring that any observed changes in cybercrime rates could
be more accurately attributed to the policy implementation rather than other confounding factors
related to internet usage.

  \item \textbf{Variable Definition}\par

 \leavevmode\hspace*{1.5em} In our DID model, the dependent variable Yit denotes the number of cybercrimes in country i
during year t. The policy variable treatedit is a binary variable: a value of 1 indicates that coun-
try had implemented the relevant policy in year, while 0 means no implementation. The time var-
iable timeit is also binary, with 1 representing the post - policy implementation period and 0 for
the pre - period.
\par
\leavevmode\hspace*{1.5em}To control for other factors that might influence cybercrime rates, we incorporated several
control variables, including GDP, internet penetration rate, mobile broadband subscribers, tertiary
education enrollment rate, legal measures, capacity development, and cooperation measures.
\par
\leavevmode\hspace*{1.5em}Given the varying magnitudes of these control variables, we normalized them using the min - max
normalization method. This normalization ensured that all variables were on a comparable scale,
preventing any single variable from disproportionately influencing the model results due to its
large magnitude.

\item \textbf{Model Specification}
\par
\leavevmode\hspace*{1.5em}The DID model was formulated as follows:
  \[
    Y_it = \alpha + \beta_{1} treatedit + \beta_{2} time_{it} + \beta_{3} (treated_{it} \times time_{it}) + \sum_{k} \gamma_k X_{kit} + \epsilon_{it}
  \]
\par
\leavevmode\hspace*{1.5em}After categorizing the policies, we utilized the Difference - in - Differences (DID) approach
to assess the causal impact of each policy theme on cybercrime rates.
\end{enumerate}

\subsection{Decoding the Policy Efficacy: LDA - DID Results Uncovered}

\begin{enumerate}
  \item \textbf{Cybercrime Criminal Legislation}
\par
\leavevmode\hspace*{1.5em}For the theme of Cybercrime Criminal Legislation, we chose Australia and Ireland as the
experimental group and the UK and New Zealand as the control group. The Diff - in - Diff effect
value was 0.151, but it was not significant ( $p>0.05$ ), suggesting that the policy in this area had
no discernible impact on cybercrime rates. Before the experiment, the Diff effect value was 0.062
( $p>0.05$ ), indicating no significant difference between the experimental and control groups.
After the experiment, the Diff effect value was 0.212  ( $p>0.05$ ), still showing no significant
difference. This lack of significance implies that, at least within the scope of our study, the existing
cybercrime criminal legislation may not be effectively curbing cybercrimes.

  \begin{table}[H]
    \caption{Summary of the DID model for Cybercrime Criminal Legislation}
    \centering
    \begin{tabular}{cccccc}
      \toprule
      time & Item & \makecell{Effected value\\Normalized number of cases} & Standard error & t & p \\
      \midrule

      \multirow{3}{*}{Before}
        & Control       & 0.009  &      &       &       \\
        & Treated       & 0.071  &      &       &       \\
        & Diff (T -- C) & 0.062  & 0.25 & 0.248 & 0.805 \\

      \multirow{3}{*}{After}
        & Control       & -0.326 &      &       &       \\
        & Treated       & -0.113 &      &       &       \\
        & Diff (T -- C) & 0.212  & null & null  & null  \\

        & Diff-in-Diff  & 0.151  & 0.107 & 1.411 & 0.168 \\
      \bottomrule
    \end{tabular}
    \begin{tablenotes}
        \footnotesize
        \item Remarks: $R^2=0.713$ adjust $R^2=0.635$.\\
        \item * p<0.05 ** p<0.01
      \end{tablenotes}
  \end{table}

  \item \textbf{Data Protection and Privacy Regulations}
\par
\leavevmode\hspace*{1.5em}As most countries have related laws, the UK (without such policies in collected data) was
chosen as the control group, and Australia (with similar internet penetration) as the experimental
group.
                                                                                                                       
\par
\leavevmode\hspace*{1.5em}The Diff - in - Diff value is 0.708, significant at the 5\% level ( $p=0.022$ ). Before the ex-
periment, the Diff value of - 0.206 is not significant ( $p>0.05$ ), indicating no difference between
groups. After the experiment, the Diff value of 0.502 is also not significant ( $p>0.05$ ).
\par
\leavevmode\hspace*{1.5em}Despite non - significant pre and post - experiment Diff values, the significant Diff - in - Diff
value implies that Data Protection and Privacy Regulations are effective in curbing cybercrimes,
likely by making data - related crimes more difficult.

  \begin{table}[H]
    \centering
    \begin{tabular}{lccccc}
      \toprule
        & \makecell{Regression\\Coefficient} & \makecell{Standard\\Error} & t & p & 95\% CI \\
      \midrule
      treated        & -0.206 & 0.236 & -0.874 & 0.396 & -0.708 $\sim$ 0.296 \\
      time           & -0.814 & 0.185 & -4.39  & 0.001 & -1.209 $\sim$ -0.419 \\
      $treated \times time$ & 0.708  & 0.278 & 2.547  & 0.022 & 0.116 $\sim$ 1.299 \\
      \bottomrule
    \end{tabular}
    \begin{tablenotes}
        \footnotesize
        \item Remarks:Explained Variable = Normalized Number of Cases\\
        \item * p<0.05 ** p<0.01
      \end{tablenotes}
  \end{table}    

\item \textbf{Cybersecurity Obligations of Network Operators}
\par
\leavevmode\hspace*{1.5em}For the theme of Cybersecurity Obligations of Network Operators, with Australia and China
as the experimental group and France and the Netherlands as the control group, the Diff - in - Diff
effect value was 0.310, demonstrating a strong positive effect. Before the experiment, the Diff
effect value was - 0.402, which might be due to pre - existing differences in cyber - security pos-
tures between the two groups. The significance of the interaction term ( P= 0.01 ) indicates that
clearly defined cybersecurity obligations for network operators are effective. These obligations
may require operators to conduct regular security audits, implement security patches promptly,
and report security incidents in a timely manner. By doing so, they can proactively prevent cyberat-
tacks and reduce the overall cybercrime rate.

  \begin{table}[H]
    \centering
    \begin{tabular}{lccccc}
      \toprule
        & \makecell{Regression\\Coefficient} & \makecell{Standard\\Error} & t & p & 95\% CI \\
      \midrule
      treated        & -0.402 & 0.126 & -3.204 & 0.003 & -0.658 $\sim$ -0.146 \\
      time           & -0.448 & 0.095 & -4.688 & 0     & -0.642 $\sim$ -0.253 \\
      $treated \times time$ & 0.31   & 0.113 & 2.75   & 0.01  & 0.080 $\sim$ 0.540 \\
      \bottomrule
    \end{tabular}
    \begin{tablenotes}
        \footnotesize
        \item Remarks:Explained Variable = Normalized Number of Cases\\
        \item * p<0.05 ** p<0.01
    \end{tablenotes}
  \end{table}    

\item \textbf{Critical Information Infrastructure Protection}
\par
\leavevmode\hspace*{1.5em}For Critical Information Infrastructure Protection, the Diff - in - Diff effect value was 0.286
showing no significant impact. However, after the experiment, the Diff effect value was 0.634,
indicating that the experimental group had a higher cybercrime rate than the control group. This
\par
\leavevmode\hspace*{1.5em}unexpected result could be because protecting critical information infrastructure is a complex task
that involves multiple stakeholders and technical challenges. The existing policies may not be
comprehensive enough to address all aspects of the protection, such as the coordination between
different sectors and the continuous evolution of cyber threats.

\item \textbf{Cybersecurity Incident Emergency Response Mechanisms}
\par
\leavevmode\hspace*{1.5em}When considering Cybersecurity Incident Emergency Response Mechanisms, the Diff - in -
Diff effect value was - 0.189, indicating no significant impact. This may suggest that the current
emergency response mechanisms are not well - designed or that there are difficulties in their im-
plementation. For example, there may be a lack of clear communication channels between different
response teams, or the response time may be too long to effectively mitigate the impact of cyber -
security incidents.

\item \textbf{Cross - Border Data Flow Rules}
\par
\leavevmode\hspace*{1.5em}For the theme of Cross - Border Data Flow Rules, with Australia and Ireland as the experi-
mental group and the UK and Singapore as the control group, the significance of the interaction
term between  treated and time ( $p=0.047$ ) suggests that the policy in this area is effective.
This could be due to the clear regulation of cross - border data flow, which may prevent cyber-
criminals from exploiting cross - border data loopholes.
\end{enumerate}

  \begin{table}[H]
    \centering
    \begin{tabular}{lccccc}
      \toprule
        & \makecell{Regression Co-\\efficient} & \makecell{Standard\\Error} & t & p & 95\% CI \\
      \midrule
      treated        & -0.402 & 0.126 & -3.204 & 0.003 & -0.658 $\sim$ -0.146 \\
      time           & -0.448 & 0.095 & -4.688 & 0     & -0.642 $\sim$ -0.253 \\
      $treated \times time$ & 0.31   & 0.113 & 2.75   & 0.01  & 0.080 $\sim$ 0.540 \\
      \bottomrule
    \end{tabular}
    \begin{tablenotes}
        \footnotesize
        \item Remarks:Explained Variable = Normalized Number of Cases\\
        \item * p<0.05 ** p<0.01
      \end{tablenotes}
  \end{table}

\par
In summary, our analysis using LDA and DID has provided valuable insights into the effec-
tiveness of different cyber - security policy themes. Data protection and privacy regulations, cy-
bersecurity obligations of network operators, and cross - border data flow rules have shown posi-
tive impacts on reducing cybercrime rates, while cybercrime criminal legislation, critical infor-
mation infrastructure protection, and cybersecurity incident emergency response mechanisms may
need further improvement. These findings can guide policymakers in formulating more effective
cyber - security policies and enhancing national cyber - security postures.

\section{MIC \& Factor Strategy: Demographics - Cybercrime Correlations}

Building on the exploration of global cybercrime distribution and the assessment of national
security policies' efficacy, we continue to explore the correlations between various national - level
demographic factors, such as GDP, Internet Penetration Rate, Active mobile - broadband subscrip-
tions, School enrollment in tertiary education, GDP per capita growth, and GNI per capita, and the
                                                                                                                     
analyze data from 20 representative countries. Additionally, factor analysis is conducted to further
clarify the relationships. The results show that different variables have varying degrees of influ-
ence on cybercrime cases across countries, and the overall average influence of variables also
differs.

\subsection{Analytical Framework: MIC Approach}

Given the complexity and non - linear nature of real - world data, traditional correlation met-
rics like Pearson or Spearman coefficients may fall short. The MIC method, renowned for its uni-
versality, fairness and symmetry, emerges as a fitting choice.\cite{9}

\subsubsection{Data Preparation}

Data on a suite of variables, including GDP, Internet Penetration Rate(IPR), Active mobile -
broadband subscriptions(AMBS), School enrollment in tertiary education(SETE), GDP per capita
growth(GDG), GNI per capita(GNC), and the Number of Cybercrime Cases, was amassed from
20 representative countries. These data were then ingested into a Pandas DataFrame, with the rel-
evant features (independent variables) and the target variable (Number of Cybercrime Cases) me-
ticulously extracted. Of course, rigorous data pre - processing was carried out to ensure data integ-
rity and suitability for analysis.

\subsubsection{MIC Computation}

For each pair of a feature variable and the Number of Cybercrime Cases (e.g., GNI per capita
and the Number of Cybercrime Cases), the scatter plots were subjected to a gridding process. By
exploring a plethora of i and j scale combinations (e.g., i = 2, j = 2, among numerous others), mul-
tiple gridding scenarios were generated. The mutual information for each scenario was computed,
and the scheme yielding the maximum mutual information was identified. This maximum value
was then normalized. Through an exhaustive search across all scale combinations, the MIC value,
representing the strongest association, was determined for each variable - target pair.

\subsection{Factor Analysis for Deeper Understanding}

In order to make the research results more convincing and provide effective reference for the
implementation of national policies, factor analysis was utilized as a supplementary method.\cite{10}
First, Bartlett's test of sphericity was carried out, and the result with a p - value of 0.000 indicated
the suitability of factor analysis for the data.
The factor analysis was then performed on variables including GDP per capita growth, GNI
per capita, Internet access, Active mobile - broadband subscriptions, and School enrollment in
tertiary education. Two factors were extracted, both having eigenvalues greater than 1. After rota-
tion, these factors explained 50.657\% and 24.605\% of the variance respectively, with a cumulative
variance explanation of 75.262\%.

A composite score was formulated as follows:
\[
  Composite\ Score = 0.673 \times Factor\ Score_1 + 0.327 \times Factor\ Score2
\]

Subsequently, step - by - step linear regression analysis was conducted using Factor 1, Factor
2, and Y(the Number of Cyber - crime Incidents (NSCI)). It was found that Factor 2 had no direct
impact on NSCI, resulting in the equation:

\[
  Y = 69.676 + 7.219 \times Factor1
\]

The calculated result of the coefficient of each variable as follows:

\begin{table}[H]
  \centering
  \caption{The Coefficient of Each Variable}
  \begin{tabular}{cccccc}
    \toprule
      & GNC & GDG & SETE & AMBS & IPR \\
    \midrule
    Factor 1 & 0.278 & -0.082 & 0.272 & -0.322 & 0.376 \\
    \bottomrule
  \end{tabular}
\end{table}


By substituting the expression of Factor Score 1 into the NSCI equation, the final relationship
was derived:
\[
  Y = 69.676 + 2.007 \times GNC - 0.592 \times GDG + 1.964 \times SETE - 2.324 \times AMBS + 2.714 \times IPR
\]

This process enabled a more in - depth exploration of the relationships between the demo-
graphic variables and the number of cyber - crime incidents, complementing the insights from the
MIC analysis.

\subsection{Results Unveiled: Variable Impact Profiles}
\subsubsection{Intra - variable Variations across Countries}

\begin{itemize}[\ding{226}]
 \item \textbf{GDP:} MIC values for GDP and cybercrime cases vary across countries. For example, Russia's
0.71 shows a strong link, perhaps due to its economic scale and structure. In contrast, Spain
and Belgium's 0.25 indicates GDP has a minor role, with other factors being more influential.

 \item \textbf{Internet Penetration Rate:} Though Internet penetration rate - cybercrime MIC values are
generally high, there are country differences. Australia's 0.97 and Singapore's 0.74 suggest
that usage environments and security measures play a role. Russia's high - activity digital
landscape may strengthen the link, while Singapore's security and user education weaken it.

 \item \textbf{Active mobile - broadband subscriptions:} MIC values for this variable differ greatly. 
 Belgium's 0.88 implies a strong connection to cybercrime, maybe due to usage patterns. 
Australia's 0.48 indicates a weaker link, likely because of good network governance and user awareness.

 \item \textbf{Tertiary Education Enrollment:} The MIC values for tertiary education enrollment and cy-
bercrime also display a wide range of variation. China has a relatively high MIC value of 0.69,
which could be related to the online behavior and technological proficiency of the highly -
educated population. Conversely, the Netherlands and Singapore have lower MIC values of
0.35, indicating that tertiary education enrollment has a less significant influence on cyber-
crime in these countries.
                                                                                                               
 \item \textbf{GDP per capita growth:} Turkey exhibited a strikingly high MIC value of 0.96, highlighting
a close connection between GDP per capita growth and cybercrime, possibly driven by the
rapid evolution of network - based economic activities during its growth phase. In Russia, the
MIC value of 0.65 was relatively lower, implying that GDP per capita growth has a less sig-
significant impact on cybercrime, possibly due to its well - established economic and cyber -
regulatory frameworks.
\end{itemize}

\subsubsection{Inter - variable Comparative Analysis}

From the average MIC values, the Internet Penetration Rate demonstrated a relatively high
average MIC value of 0.849 with the Number of Cybercrime Cases, underscoring its strong overall
association. As the primary medium for cybercrime, the Internet's penetration level directly shapes
the landscape for cybercriminal activities. Active mobile - broadband subscriptions also showed a
substantial average MIC value of 0.682, indicating that the proliferation of mobile broadband con-
tributes to the cybercrime risk. In contrast, GDP had the lowest average MIC value of 0.3215,
suggesting that the mere magnitude of GDP has a limited impact on cybercrime, with network -
related and economic structural factors likely playing more complex roles.

  \begin{figure}[H]
      \centering
      \includegraphics[width=10cm]{figure9.png}
      \caption{MIC values for each variable}
  \end{figure}

In summary, different variables exhibit varying degrees of influence across countries, and
their average impact levels also diverge. Variables like the Internet Penetration Rate, Active
mobile - broadband subscriptions, and GDP per capita growth show relatively strong correla-
tions with cybercrime, while GDP and tertiary education enrollment have more subdued effects,
which is similar to those of factor analysis. These findings offer valuable insights into the intricate
interplay between national - level characteristics and cybercrime, guiding the formulation of tar-
geted cyber - security strategies. However, further research is imperative to explore the underlying
mechanisms in greater depth and account for additional complex factors to more effectively com-
bat cybercrime.

\section{Sensitivity Analysis}

Security Policies vs distribution of cybercrimes Differences-in-Differences Model and Maximal
 Information Coefficient Model are our basic models, so we con-ducted sensitivity analysis
on these two models.                                                                                                       

In the formula of the DID model for the comparison of security policies and crime distribution,
treatedtime is the interaction term, which is the core of the DID model, and its coefficient $\beta$ 
measures the causal effect of policy implementation on the number of cybercrimes.
We selected the fourth group - Cybersecurity Obligations of Network Operators for analysis, al-
lowing $\beta$ to fluctuate by 5\% ($0.286\pm 0.0143$), and plotted the changes in the p-value (which reflects
the effect value of significance), and the Diff-in-Diff effect value (before and after the experiment)
with the change in $\beta$, as shown in the figure below. It is easy to see that the changes in the three
are stable with the fluctuation of $\beta$ (p changed within a range of $0.229\pm 0.0005$). Calculations show
that the change amplitude does not exceed 5\%, which meets our stability expectations and proves
that our data selection is reasonable. The model has passed the sensitivity analysis.

  \begin{figure}[H]
      \centering
      \includegraphics[width=10cm]{figure9.png}
      \caption{Relationship between $\beta$, P, DID(before) and DID(after)}
  \end{figure}

What's more, many institutions, especially investment firms, are reluctant to report hacker
attacks due to concerns that disclosing such attacks could damage their reputation, leading to cus-
tomer loss and business impairment. Instead, they tend to quietly pay the ransom. Therefore, to
verify the robustness and stability of the MIC model, we added random noise to the data to simulate
the potential impact of hackers and then calculated the MIC values for the data that might contain
outliers and observed the changes in these values. After repeating these steps multiple times, we
calculated the mean and standard deviation of the MIC values to evaluate the model's stability.

Taking Belgium as an example, we randomly added noise to the data multiple times and then
calculated the average MIC values of six indicators, which were compared with the original data.
The results are presented in the WiFi infographic. If the end of the ring exceeds the gray asymptote,
the new MIC value is higher than the previous one.As expected, the new average MIC values of
the six indicators are all stably distributed near the gray dotted line, indicating that the robustness
and stability of the MIC model are quite remarkable.

\section{Model Evaluation and Further Discussion}
\subsection{Strengths}
\begin{enumerate}
  \item \textbf{Comprehensive Data - Driven Analysis}

  \par
\leavevmode\hspace*{1.5em}The use of multiple data sources and indices, such as the National Cyber Security Index
(NCSI), Cybersecurity Exposure Index (CEI), and Legal Measures from the Global Cybersecurity
Index, provides a comprehensive view. These data sources capture different dimensions of cyber-
crime, from a country's preparedness against threats (NCSI) to the frequency of malicious attacks
and cybersecurity commitment (CEI), and the legal framework for prosecution. Calculating the
network threat reporting rate further enriches the dataset. This multi - source data approach enables
a more in - depth understanding of cybercrime distribution patterns globally.
  \item \textbf{Advanced Analytical Methods}
  \begin{itemize}[\ding{226}]
  \item   We creatively came up with the KDMF model, which integrated these four sub - models
and provided a holistic view of cybercrime. It enables a comprehensive analysis of cy-
bercrime distribution, policy effectiveness, and the influence of demographics, offering
a solid foundation for data - driven development and refinement of national cybersecurity
policies.

 \item When evaluating the impact of national security policies on cybercrime rates, the DID
model proves indispensable. By comparing an experimental group of countries imple-
menting a specific policy with a control group of similar - internet - penetration countries
lacking that policy, while controlling for factors like GDP and internet penetration rate,
it can precisely isolate the policy's causal effect. This allows researchers to pinpoint
which policy components are truly effective in curbing cybercrime, whether it's strength-
ening cybercrime criminal legislation or enhancing data protection regulations.

 \item The combination of MIC and factor analysis offers a more comprehensive analysis
framework. Factor analysis provides a high - level view of the data structure, highlighting
the underlying factors that drive the relationships among variables. MIC, in turn, vali-
dates and supplements the results of factor analysis. By using MIC to cross - check the
relationships identified by factor analysis, we can ensure that all possible relationships,
whether simple linear ones or complex non - linear ones, are thoroughly explored. This
mutual verification not only enhances the accuracy of our analysis but also deepens our
understanding of the complex interplay between national demographics and cybercrime
cases.
  \end{itemize}
\end{enumerate}

\subsection{Weaknesses}
\begin{enumerate}
  \item \textbf{Data - related Limitations}
The calculation of the network threat reporting rate depends on data from THE VERIS COM-
MUNITY DATABASE (VCDB) and the World Bank. Since VCDB only includes publicly dis-
closed data breaches, it may lead to an under - estimation of the actual number of cyber threats.
  \item \textbf{Model - related Constraints} 
  \begin{itemize}[\ding{226}]
  \item \textbf{K - clustering} analysis may oversimplify the complex relationships between different
factors contributing to cybercrime distribution. It assumes distinct and homogeneous
clusters, while in reality, there may be overlapping characteristics and complex interac-
tions between countries within and across clusters.
                                                                                                                  
  \item The application of the DID model faces challenges in selecting appropriate experimental
and control groups. Identifying countries with exactly the same or similar internet pene-
tration rates and other characteristics is difficult, and there may be unobserved factors
that differ between the two groups, confounding the results. Cultural differences, political
stability, or law enforcement efficiency, which can influence cybercrime rates, may not
be accounted for in the model.

  \item Although the MIC method can capture non - linear relationships, interpreting the MIC
values can be complex. A high MIC value does not necessarily imply causation, and the
values may be affected by data scale and range. Different data transformations can lead
to different MIC values, making cross - study comparisons difficult.
  \end{itemize}
\end{enumerate}

\subsection{Further Discussion}
\begin{enumerate}
  \item \textbf{Augmenting Data Quality: Expansion and Standardization}\par
\leavevmode\hspace*{1.5em}Future studies should cast a wider net to gather data from diverse sources. This includes in-
corporating data from multiple international organizations, private - sector reports, and compre-
hensive cyber threat databases. Data standardization is equally vital. Harmonizing data collection
and reporting methods across different sources ensures data comparability.
  \item \textbf{Policy - Oriented Exploration}\par
\leavevmode\hspace*{1.5em}A more comprehensive evaluation of national cybersecurity policies is imperative based on
the current analysis. Future research should delve into how different policy components interact
with each other and with demographic factors. For example, understanding how the combination
of data protection regulations and cybercrime criminal legislation impacts cybercrime rates across
different economic and social contexts can offer valuable insights for policymakers.
In addition, given the disparities in cybercrime distribution and policy effectiveness among
countries, policies must be tailored to specific national contexts. Future research can leverage the
analysis results to develop more targeted and effective cybersecurity policies. This involves con-
sidering each country's unique demographic, economic, and social characteristics, ensuring that
policies are not only relevant but also practical and impactful in combating cybercrime.
\end{enumerate}                                                                                                   

\newpage
\begin{thebibliography}{99}
\bibitem{1} Morgan, S. (2020, November 13). Cybercrime To Cost The World \$10.5 Trillion Annually By
2025. Cybercrime Magazine.
\bibitem{2} Verizon RISK Team. (n.d.). The VERIS Community Database (VCDB). VERIS Frame-
work. \\ https://verisframework.org/vcdb.html
\bibitem{3} e - Governance Academy. (2016, May 23). The National Cyber Security Index gives govern-
ments a tool for developing cyber security. e - Governance Academy. \\ https://ega.ee/news/the-national-cyber-security-index-gives-governments-a-tool-for-developing-cyber-security/
\bibitem{4} Password Managers. (n.d.). Cybersecurity Exposure Index. Password Managers. \\ https://passwordmanagers.co/cybersecurity-exposure-index/
\bibitem{5} International Telecommunication Union. (2024). Global Cybersecurity Index 2024 (5th Edi-
tion).
\bibitem{6} Lloyd, S. P. (1982). Least squares quantization in PCM. IEEE Transactions on Information
Theory, 28(2), 129-137. https://doi.org/10.1109/TIT.1982.1056489
\bibitem{7} Blei, D. M., Ng, A. Y., \& Jordan, M. I. (2003). Latent Dirichlet Allocation. Journal of Machine
Learning Research, 3, 993-1022.
\bibitem{8} Card, D., \& Krueger, A. B. (1994). Minimum wages and employment: A case study of the fast-
food industry in New Jersey and Pennsylvania. The American Economic Review, 84(4), 772-793.
\bibitem{9} Reshef, D. N., Reshef, Y. A., Finucane, H. K., Grossman, S. R., McVean, G., Turnbaugh, P.
J., ...\& Sabeti, P. C. (2011). Detecting novel associations in large data sets. Science, 334(6062),
1518-1524.
\bibitem{10} Hair, J. F., Anderson, R. E., Tatham, R. L., \& Black, W. C. (1998). Multivariate Data Analysis
(5th ed.). Prentice Hall.
\end{thebibliography}

\end{document}
