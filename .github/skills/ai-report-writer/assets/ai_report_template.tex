% AI Tool Usage Report Template for MCM/ICM
% This template follows COMAP's official requirements
% Place this report AFTER the 25-page solution as an appendix

\newpage
\appendix

% ============================================================================
% REPORT ON USE OF AI TOOLS
% ============================================================================

\section*{Report on Use of AI Tools}
\addcontentsline{toc}{section}{Report on Use of AI Tools}

% ----------------------------------------------------------------------------
% BASIC INFORMATION
% ----------------------------------------------------------------------------

\subsection*{Basic Information}
\begin{itemize}
    \item \textbf{Contest}: MCM/ICM 2026
    \item \textbf{Team Number}: [YOUR TEAM NUMBER]
    \item \textbf{Problem}: [A/B/C/D/E/F]
    \item \textbf{Report Language}: English
    \item \textbf{Date}: [Submission Date]
\end{itemize}

% ----------------------------------------------------------------------------
% SECTION 1: USAGE OVERVIEW
% ----------------------------------------------------------------------------

\subsection*{1. Usage Overview}

\begin{table}[H]
\centering
\caption{Summary of AI Tools Used}
\label{tab:ai-overview}
\begin{tabular}{p{3cm}p{2.5cm}p{2cm}p{3cm}p{3cm}}
\toprule
\textbf{Tool/Model} & \textbf{Version/Date} & \textbf{Type} & \textbf{Purpose} & \textbf{Location} \\
\midrule
% Example entries - replace with your actual tools
OpenAI ChatGPT GPT-4 & 2026-01-29 & LLM & Polishing introduction & Section 1 \\
Claude 3.5 Sonnet & 2026-01-30 & LLM & Debugging Python code & code/model.py \\
DeepL Translator & 2026-01-29 & Translation & Literature translation & Literature notes \\
GitHub Copilot & VSCode 2026-01 & Code Copilot & Auto-complete functions & code/data\_cleaner.py \\
% Add more rows as needed
\bottomrule
\end{tabular}
\end{table}

% ----------------------------------------------------------------------------
% SECTION 2: DETAILED RECORDS OF LLM / GENERATIVE AI
% ----------------------------------------------------------------------------

\subsection*{2. Detailed Records of LLM / Generative AI}

% ---- Tool 1: Example LLM ----
\subsubsection*{2.1 Tool/Model Information}
\textbf{Tool}: OpenAI ChatGPT GPT-4 \\
\textbf{Version}: Accessed on 2026-01-29 via web interface (chat.openai.com) \\
\textbf{Purpose}: Polishing the introduction section for clarity and academic tone

\subsubsection*{2.2 Interaction Log}

\paragraph{Interaction 1: Introduction Polishing}
\textbf{Time}: 2026-01-29 14:30 \\

\textbf{Query (Exact Wording)}:
\begin{quote}
\textit{``Please polish the following paragraph for academic tone and clarity: [Insert your actual prompt here]''}
\end{quote}

\textbf{Output (Complete Response)}:
\begin{quote}
\textit{[Insert full AI response here. Do not summarize. Include complete text.]}
\end{quote}

\textbf{Verification and Revision}:
\begin{itemize}
    \item \textbf{Verified}: [Specific verification steps, e.g., "Checked that no new factual claims were introduced. Confirmed all terminology matches our model definitions."]
    \item \textbf{Revised}: [Specific changes, e.g., "Changed 'utilize' back to 'use' for simplicity. Removed one sentence that was too verbose."]
    \item \textbf{Final Usage}: [Percentage or description, e.g., "Used approximately 70\% of the AI output. The final paragraph in Section 1 is a combination of AI suggestions and our original wording."]
\end{itemize}

\paragraph{Interaction 2: [Description]}
\textbf{Time}: [YYYY-MM-DD HH:MM] \\

\textbf{Query (Exact Wording)}:
\begin{quote}
\textit{``[Your exact prompt]''}
\end{quote}

\textbf{Output (Complete Response)}:
\begin{quote}
\textit{[Full AI response]}
\end{quote}

\textbf{Verification and Revision}:
\begin{itemize}
    \item \textbf{Verified}: [Specific checks]
    \item \textbf{Revised}: [Specific changes]
    \item \textbf{Final Usage}: [How much used]
\end{itemize}

% Add more interactions as needed
% Repeat the "Interaction N" block for each LLM interaction

% ---- Tool 2: Another LLM (if used) ----
\subsubsection*{2.3 Tool/Model Information}
\textbf{Tool}: Claude 3.5 Sonnet \\
\textbf{Version}: Accessed on 2026-01-30 via web interface (claude.ai) \\
\textbf{Purpose}: Debugging Python code for optimization model

\subsubsection*{2.4 Interaction Log}

\paragraph{Interaction 1: Debugging ValueError}
\textbf{Time}: 2026-01-30 15:45 \\

\textbf{Query (Exact Wording)}:
\begin{quote}
\textit{``[Your exact debugging prompt]''}
\end{quote}

\textbf{Output (Complete Response)}:
\begin{quote}
\textit{[Full AI response]}
\end{quote}

\textbf{Verification and Revision}:
\begin{itemize}
    \item \textbf{Verified}: [Testing performed]
    \item \textbf{Revised}: [Code modifications]
    \item \textbf{Final Usage}: [Percentage of AI suggestion used]
\end{itemize}

% Add more tools and interactions as needed

% ----------------------------------------------------------------------------
% SECTION 3: TRANSLATION TOOLS
% ----------------------------------------------------------------------------

\subsection*{3. Translation Tools}

% Note: Full input text is NOT required for translation tools per COMAP policy

\textbf{Tool/Model}: DeepL Translator (Web interface, deepl.com) \\
\textbf{Version/Date}: Accessed on 2026-01-29 \\

\textbf{Usage Statement}: 
We used DeepL to translate 5 Chinese academic paper abstracts (approximately 2000 words total) to understand relevant prior work. The source papers were:
\begin{itemize}
    \item Zhang et al. (2023) - [Chinese title] → [English translation]
    \item Li et al. (2022) - [Chinese title] → [English translation]
    % Add more as needed
\end{itemize}

\textbf{Proofreading Notes}: 
All translations were manually proofread by native English speakers on our team. We verified technical terms (e.g., ``非线性规划'' $\rightarrow$ ``nonlinear programming'') against standard English-Chinese mathematical dictionaries. We confirmed logical consistency between original Chinese text and translated English versions. Key technical terms were cross-referenced with English literature to ensure proper terminology.

% Add more translation tools if used (e.g., Google Translate, Baidu Fanyi)

% ----------------------------------------------------------------------------
% SECTION 4: CODE COPILOTS / AUTO-COMPLETE / MATHEMATICAL SOFTWARE AI
% ----------------------------------------------------------------------------

\subsection*{4. Code Copilots / Auto-complete / Mathematical Software AI}

\textbf{Tool/Model}: GitHub Copilot (VSCode extension, version 1.150.0) \\
\textbf{Version/Date}: Used throughout competition (2026-01-29 to 2026-02-01) \\

\textbf{Purpose}: Auto-complete Python functions for data preprocessing and model implementation \\

\textbf{Usage Location}: 
\begin{itemize}
    \item \texttt{code/data\_cleaner.py} (lines 23-45, 78-92)
    \item \texttt{code/optimizer.py} (lines 156-203)
    \item \texttt{code/visualization.py} (lines 34-67)
\end{itemize}

\textbf{Verification Notes}: 
\begin{itemize}
    \item \textbf{Testing}: All Copilot-generated functions were tested with unit tests on sample data. We verified correctness by comparing outputs with manually calculated expected results.
    \item \textbf{Modifications}: We modified variable names for clarity and consistency with our codebase conventions. We added error handling (try-except blocks) that Copilot did not generate. We added comprehensive docstrings explaining parameters, return values, and edge cases.
    \item \textbf{Human Oversight}: All Copilot suggestions were reviewed by team members before integration. We rejected approximately 30\% of suggestions that did not fit our needs or contained logical errors.
    \item \textbf{Final Assessment}: The final code is approximately 60\% human-written and 40\% Copilot-assisted. All critical algorithms and model logic were designed by humans; Copilot primarily assisted with boilerplate code and standard data manipulation patterns.
\end{itemize}

% Add more code assistance tools if used (e.g., Cursor, Tabnine, Amazon CodeWhisperer)

% If you used Wolfram Alpha AI mode or other math software AI:
% \textbf{Tool/Model}: Wolfram Alpha AI Mode \\
% \textbf{Purpose}: [Describe usage] \\
% \textbf{Verification}: [Describe how results were verified]

% ----------------------------------------------------------------------------
% SECTION 5: INTEGRITY, VERIFICATION, AND RESPONSIBILITY STATEMENT
% ----------------------------------------------------------------------------

\subsection*{5. Integrity, Verification, and Responsibility Statement}

We, Team [YOUR TEAM NUMBER], declare that:

\begin{enumerate}
    \item \textbf{Verification of AI Content}: All AI-generated content has been thoroughly verified for accuracy, consistency with our models and data, and alignment with our problem interpretation. We have not blindly accepted AI outputs without critical evaluation.
    
    \item \textbf{Citation Authenticity}: We have checked all citations and references for authenticity. We used Google Scholar, Web of Science, and direct journal website searches to confirm that every cited work exists and is correctly attributed. No ``hallucinated'' or fabricated citations have been included in our solution.
    
    \item \textbf{Human-Led Critical Decisions}: All critical decisions in our solution were made by human team members, including:
    \begin{itemize}
        \item Problem interpretation and modeling approach selection
        \item Model assumptions and justifications
        \item Result interpretation and sensitivity analysis
        \item Conclusions and recommendations
        \item Creative insights and novel contributions
    \end{itemize}
    
    \item \textbf{Full Responsibility}: We take full responsibility for the accuracy and integrity of our submission. We understand that any errors, including those originating from AI-generated content, are our responsibility.
    
    \item \textbf{Complete Disclosure}: This report represents a complete and honest disclosure of all AI tools used in our solution. We have not omitted any AI use, and we have provided detailed verification notes for all significant AI interactions.
    
    \item \textbf{Academic Integrity}: We affirm that our work complies with COMAP's academic integrity policies and AI usage guidelines. We have used AI as a productivity tool to enhance our human work, not as a replacement for human creativity and judgment.
\end{enumerate}

\vspace{1em}

\noindent
\textbf{Team Members}:
\begin{itemize}
    \item [Team Member 1 Name]
    \item [Team Member 2 Name]
    \item [Team Member 3 Name]
\end{itemize}

\vspace{1em}

\noindent
\textbf{Date}: [Submission Date] \\
\textbf{Team Advisor}: [Advisor Name, if applicable]

% ----------------------------------------------------------------------------
% END OF REPORT
% ----------------------------------------------------------------------------

% Optional: Add a page break before code appendices
\newpage
